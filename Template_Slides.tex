%\documentclass[aspectratio=169,11pt,professionalfonts,handout]{peqpresentation}
\documentclass[aspectratio=169,11pt,professionalfonts,presentation]{peqslides}

% For printing, use the "handout" option

\usepackage[brazil]{babel}

% PEQ Presentation default environments:
% nblock[tcolorbox options]{title}  -> Default theme blocks
% alblock[tcolorbox options]{title} -> Alert blocks
% nfitbox[tcolorbox options]{text}  -> Inner text blocks (it can be put inside another block)


%%Packages here
\usepackage{lipsum}


%%Table of contents before each section and subsection
\AtBeginSubsection[]{
  \begin{frame}<beamer>{}
    \tableofcontents[currentsection,currentsubsection]
  \end{frame} } 

%%Title page configuration
\title[Evento]{Título}
\author[Ataíde Neto]{\textbf{Ataíde Souza Andrade Neto}\\[.5ex] }
\date[Titulo curto]{\scriptsize Data: \today}


%%Presentation
\begin{document}
\titlepage

	\begin{frame}
		\tableofcontents
	\end{frame}
	
\section{Primeira seção}
\subsection{Primeira subseção}
	\begin{frame}{Exemplo}
	
		\begin{alblock}{Documentação}
			Todos os blocos são criados com o pacote \textbf{tcolorbox}
		\end{alblock}
		
		\begin{tcbraster}[raster columns=2,raster equal height]
			\uncover<2->{\begin{nblock}[halign title = center]{Modelo 1}
				\begin{align*}
				&\frac{du}{dt}=1-u-Du\exp(v)\\[2ex]
				&\frac{dv}{dt}=-v +BDu\exp(v)-\beta v
				\end{align*}
			\end{nblock}}
			\uncover<3->{\begin{nblock}[halign title = center]{Modelo 2}
				\begin{align*}
				&\frac{du}{dt}=1-u-DuR\\[2ex]
				&\frac{dv}{dt}=-v +BDuR-\beta v\\[2ex]
				&0=R-\exp(v)
				\end{align*}
			\end{nblock}}
		\end{tcbraster}
	
	\end{frame}

\section{Segunda seção}
\subsection{Primeira subseção}

	\begin{frame}
		\nfitbox[height=1cm,halign = flush center,colback=black!10!white]{\large\color{vermelho}{\textbf{Programa de Engenharia Química}}}
		\begin{tcbraster}[raster columns=2,raster equal height]
			
			\begin{nblock}{A Coppe}
				\begin{outline}
					\1 Instituição de ensino
					\2 Pós graduação em engenharia
					\2 Mais de 50 anos
					\1 PEQ Conceito 7	
				\end{outline}
			\end{nblock}
			\begin{alblock}[colback=white, notitle]{}
				\begin{figure}
					\includegraphics[width=0.8\textwidth]{coppe-logo.jpg}
					\caption{Logo}
				\end{figure}
			\end{alblock}
		\end{tcbraster}
	
	\end{frame}

\subsection{Segunda subseção}
		\begin{frame}
	\uncover<1->{\begin{alblock}[height=\textheight,colback = white]{Decomposição em valores característicos}\vspace{-.25cm}
			\uncover<1>{\nfitbox[height=1.85cm,colback=black!02!white]{	Um valor característico (ou autovalor) é um número $ \lambda\in\mathbb{C} $ tal que
					\begin{equation}
					\mathbf{Jv}_i=\lambda_i\mathbf{v}_i,\qquad i=1,2,\ldots,N
					\end{equation}}}\vspace{-0.2cm}
			\uncover<2>{\nfitbox[height=1.85cm,colback=black!02!white]{O problema de autovalor é formulado da seguinte maneira:
					\begin{equation}
					\det(\mathbf{J}-\lambda \mathbf{I})=0
					\end{equation}}}\vspace{-0.2cm}
			\uncover<3>{\nfitbox[height=3.25cm,colback=black!02!white]{	A matriz $ \mathbf{J} $ pode então ser fatorada em:
					\begin{equation}
					\mathbf{J=P}\boldsymbol{\Lambda}\mathbf{P}^{-1}
					\end{equation}\vspace{-0.75cm}
					\begin{outline}
						\1 $ \mathbf{P}=[v_1\ v_2\ \ldots\ v_N ] $\\[1.5ex]
						\1 $ \boldsymbol{\Lambda}=\text{diag}(\lambda_1,\lambda_2,\ldots,\lambda_N) $
			\end{outline}}}
	\end{alblock}}
	
\end{frame}


\end{document}